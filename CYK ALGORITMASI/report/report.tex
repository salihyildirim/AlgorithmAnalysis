\documentclass[a4paper]{article}

\usepackage{graphicx}
\usepackage{hyperref}
\usepackage[utf8]{inputenc}
\usepackage[tmargin=1in, rmargin=1.25in, bmargin=1in, lmargin=1.25in]{geometry}
\usepackage{pdfpages}
\usepackage{listings}
\usepackage{float}
\usepackage{color}

\definecolor{codegreen}{rgb}{0,0.6,0}
\definecolor{codegray}{rgb}{0.5,0.5,0.5}
\definecolor{codepurple}{rgb}{0.58,0,0.82}
\definecolor{backcolour}{rgb}{0.96,0.96,0.94}

\lstdefinestyle{mystyle}{
	backgroundcolor=\color{backcolour},   
	commentstyle=\color{codegreen},
	keywordstyle=\color{magenta},
	numberstyle=\tiny\color{codegray},
	stringstyle=\color{codepurple},
	basicstyle=\footnotesize,
	breakatwhitespace=false,         
	breaklines=true,                 
	captionpos=b,                    
	keepspaces=true,                 
	numbers=left,                    
	numbersep=5pt,                  
	showspaces=false,                
	showstringspaces=false,
	showtabs=false,                  
	tabsize=4
}

\lstset{style=mystyle}
\usepackage{wrapfig}

\usepackage{ptext}
\usepackage{lipsum}

\usepackage{xepersian}
\settextfont[Scale=1.1]{XB Yas}
\setlatintextfont{Arial}
\setdigitfont{Arial}


\title{تحلیل الگوریتم \lr{cyk}}
\author{محمد مهدی حیدری\\۹۴۳۱۳۰۶}
\begin{document}
	\maketitle
	در فایل 
	\lr{cyk\_main.py}
	چند تابع کمکی برای خواندن ورودی و گرامر از فایل متنی نوشته شده. در انتهای هر فایل باید یک خط خالی وجود
	داشته باشد. سمت راست و چپ هر قانون گرامر باید با $\rightarrow$ جدا شده باشد و نتایج آن می‌تواند با $|$،
	\lr{or}
	شود. یک تابع نیز برای نمایش نتیجه الگوریتم نوشته شده که در ابتدای هر خط طول زیرمجموعه‌های مورد بررسی از 
	ورودی نوشته شده است. نتیجه‌ی موجود در هر خانه با 
	\{ \}
	 مشخص شده و اگر خالی باشد به جای آن ــ گذاشته شده.
	در تابع مربوط به الگوریتم ابتدا ترمینال‌های موجود در ورودی در نظر گرفته می‌شوند. قواعدی که منجر به تولید 
	ترمینال می‌شوند به طور جداگانه ذخیره شده‌اند و فقط در مرحله اول مورد نیاز هستند. برای هر ترمینال در ورودی
	متغیری در را پیدا می‌کنیم که در سمت چپ یک قاعده که آن ترمینال را تولید می‌کند، حضور داشته باشد. در مرحله بعد
	زیر رشته‌های دوتایی را بررسی می‌کنیم. برای رخ دادن هرکدام باید در سمت راست قواعدی که دو متغیر تولید می‌کنند
	 به دنبال آنها بگردیم و متغیر سمت چپ را ذخیره کنیم‌. در مراحل بعدی برای هر زیررشته ورودی چک می‌کنیم که تحت
	 چه حالت‌هایی قابل ساخته شدن هستند. مثلا 
	 \lr{aba}
	 از ترکیب
	 \lr{ab + a}
	 یا
	 \lr{a + ba}
	 تولید می‌شود که نتایج هر یک در جدول موجود است و با یک تابع کمکی تمام حالت‌های ممکن را می‌سازیم و باز به دنبال
	 حالت‌های رخداد آن می‌گردیم در مرحله آخر اگر متغیر آغازی در نتیجه وجود داشته باشد یعنی می‌توانیم با شروع از
	 آن و استفاده از قواعد موجود در گرامر، رشته داده شده را بسازیم.
	 \newpage
	 
	  \begin{latin}
	  	\lstinputlisting[language=Python]{../cyk_main.py}
	  \end{latin}
  	برای مشاهده کد می‌توانید به 
  	\href{https://github.com/mmheydari97/automata-cyk}{اینجا}
  	مراجعه کنید.
\end{document}
